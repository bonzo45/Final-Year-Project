% book example for classicthesis.sty
\documentclass[
  % Replace twoside with oneside if you are printing your thesis on a single side
  % of the paper, or for viewing on screen.
  oneside,
  11pt, a4paper,
  footinclude=true,
  headinclude=true,
  cleardoublepage=empty
]{scrbook}

\usepackage{lipsum}
\usepackage[linedheaders,parts,pdfspacing]{classicthesis}
\usepackage{amsmath}
\usepackage{amsthm}
\usepackage{acronym}
\usepackage[nottoc,numbib]{tocbibind}

\usepackage{geometry}
\geometry{margin=1in}
\linespread{1.3}

\title{INTERIM REPORT\\Visualising uncertainty in super resolution fetal MRI images.}
\author{by\\Sam Esgate\\\\supervised by\\Bernhard Kainz}

\begin{document}

\maketitle

% This should list the main chapters and (sub)sections of your report. 
% Choose self-explanatory chapter and section titles and use double spacing for clarity.
% If possible you should include page numbers indicating where each chapter/section begins.
% Try to avoid too many levels of subheading - three is sufficient.

%*******************************************************
% Table of Contents
%*******************************************************
\pdfbookmark[1]{\contentsname}{tableofcontents}

\setcounter{tocdepth}{3} % <-- 2 includes up to subsections in the ToC
\setcounter{secnumdepth}{3} % <-- 3 numbers up to subsubsections

\tableofcontents 

%*******************************************************
% List of Figures and of the Tables
%*******************************************************
\iffalse
%*******************************************************
% List of Figures
%*******************************************************    
\pdfbookmark[1]{\listfigurename}{lof}
\listoffigures

%*******************************************************
% List of Tables
%*******************************************************
\pdfbookmark[1]{\listtablename}{lot}
\listoftables
  
%*******************************************************
% List of Listings
%******************************************************* 
\pdfbookmark[1]{\lstlistlistingname}{lol}
\lstlistoflistings 
   
%*******************************************************
% Acronyms
%*******************************************************
\pdfbookmark[1]{Acronyms}{acronyms}
\chapter*{Acronyms}
\begin{acronym}[UML]
    \acro{DRY}{Don't Repeat Yourself}
    \acro{API}{Application Programming Interface}
    \acro{UML}{Unified Modeling Language}
\end{acronym} 
\fi


\chapter{Introduction}
MRI is very suited to imaging the developing fetus due to its impressive contrast and lack of harmful side effects. However, limitations in resolution make imaging what can often be a very small target volume difficult. Super resolution is a technique to mitigate this that constructs a higher resolution image from many lower resolution images. Unfortunately super resolution introduces some uncertainty which may not be obvious from looking at the reconstructed image.

Each pixel in the reconstructed image is a weighted combination of pixels in the input images. The confidence we have for a reconstructed pixel will vary depending on the amount of information we have for that region in the low resolution images. Crucially when using these images for diagnostic purposes it is important to be aware of the uncertainty in each region.

The purpose of this project is to develop visualisations that can help a radiologist understand this uncertainty. A number of different visualisations will be tried and then compared for effectiveness to see what works well and what does not. In addition it should be possible to use the uncertainty information to guide the process of scanning. MRI is an iterative procedure where previous scans are used to decide where next to scan for best effect. Incorporating this uncertainty information at scan time might be able to guide the scanning process to minimize uncertainty and produce better images.

Hopefully the findings of this project will be able to guide the incorporation of these techniques in clinical use.

\chapter{Background}

\section{Medical Imaging}
The field of medical imaging came into existence with the discovery of x-rays in 1895 by Wilhelm R\"{o}ntgen\cite{rontgen}. This meant that for the first time it was possible to peek into the human anatomy without having to physically open it up. This revolution came at a cost however as at the time the dangers of high doses of radiation were not understood and many of the pioneers died as a result of exposure\cite{xraydeath}.

Radiology, the use of ionizing radiation to image objects, was greatly refined over the years; contrast agents were developed which made fields like angiography (imaging of blood vessels) possible\cite{infinityhistory} and the doses used were refined to dramatically decrease the associated risks. X-rays have been used successfully in medical imaging ever since and in the 70s the invention of the CT scanner allowed them to be used to produce 3D images of the body.

It was not until the 1980s that magnetic resonance imaging (MRI) was first used for medical diagnosis. MRI is based on the principles of nuclear magnetic resonance (NMR), which had been used in chemical analysis since the 40s\cite{bshr:mallard}.

MRI can be used to create anatomical images by examining the varying water content in different parts of the body. More recently MRI has been used to create functional images of the brain by using the same principals to detect blood flow\cite{fmri}. Unlike CT, which uses x-rays, MRI does not require the use of any ionizing radiation to perform the scan.

As with all medical imaging techniques the benefits and risks need to be weighed up before it is performed. When it comes to the imaging of fetuses the risks are even higher and so in general CT scans tend not to be used where possible due to radiation involved. MRI, like ultrasound, is not associated with known adverse fetal effects.\cite{pregnancyimagingguidelines}

\subsection{MRI (Magnetic Resonance Imaging)\cite{howmriworks}}

MRI is a medical imaging technique that uses a strong magnetic field and radio waves to produce images of the body. MRI can be used to produce 2D, 3D and even 4D images and has the advantage that it doesn’t use any ionizing radiation during the scanning process.

MRI is able to detect the amount of water that is contained in different types of tissue in the body by exploiting a property of protons called spin. Each water molecule in the body contains two hydrogen atoms and the nucleus of each is simply a proton.

MRI applies a strong uniform magnetic field to the area being imaged which causes the spins of the protons to line up with the direction of the field. Some of the protons line up in the same direction, with low energy, and some line up opposing the magnetic field, which have high energy.

When a radio wave is then applied some of the lower energy protons flip to become higher energy and oppose the magnetic field, when the radio waves are stopped these protons then flip back and emit a radio wave that can be detected.

The resonant frequency of radio wave that is required to excite the protons depends on the strength of the magnetic field. This property allows an area to be scanned by applying a gradient magnetic field and varying the radio frequency.

There are a number of different variations of MRI. T1-weighted scans are best for visualising fat, calcification, hemorrhage, thyroid, liver and bone marrow whereas T2-weighted scans accentuate water content.
There are also a number of different ways of acquiring the scan. SSFSE (Single Shot Fast Spin Echo) works by acquiring each slice of the image in sequence whereas 2DGRE (Gradient Echo) acquires all of the slices simultaneously.

If the contrast that can be achieved with MRI is insufficient for the imaging then a contrast agent, such as gadolinium, can be administered to enhance the quality of the image.

\subsection{Fetal MRI\cite{fetalmri}}

Since MRI, like US (ultrasound), is a non-invasive procedure that doesn’t use ionizing radiation it is ideal for use in fetal imaging. While US remains the most common imaging modality used during pregnancy MRI has a number of advantages; in particular it has better contrast in soft tissue which allows individual organs to be better differentiated. US however remains the cheaper of the two and is unlikely to be replaced as the main fetal imaging tool in the near future.

There are no known side effects associated with using MRI, however it is currently recommended to wait until the 17th or 18th WG (week of gestation) before performing MRI. This is partially a precautionary measure but also before then the size of the fetus and amount of movement limits the usefulness of the scan.

\subsection{Super Resolution\cite{superresolution1}\cite{superresolution2}}
One of the challenges with using MRI for fetal imaging the resolution that is achievable with current scanners. A typical slice thickness when performing a scan on a fetal brain for example is 3mm, which is set to provide a good balance between resolution and noise. Comparing this to the brain, which may only be 5cm, limits the diagnostic potential.

Super resolution is a technique that is designed to address this issue. The idea behind super resolution is to construct a higher resolution image by combining information from multiple lower resolution images. A similar approach is used in stereoscopic 3D where the information from two images taken from different perspectives are combined to create one 3D image.

In the case of super resolution two (or more) images are taken such that the translation between them known to subpixel accuracy. Having images where the pixels are interleaved in this way allows a higher resolution image to be constructed by interpolation of these images.

\begin{verbatim}
	<three images: red, green, blue>
	<become registered>
	<add sample grid>
	<high resolution interpolated from the three>
\end{verbatim}

\subsubsection*{Intuitive Super Resolution Overview}
The most simple form of super resolution follows three steps:

\begin{enumerate}
	\item Registration
	\item Interpolation
	\item Post-processing
\end{enumerate}

The registration step is where the translation between each image is established. Crucially this must be known to subpixel accuracy and must not be an integer number of pixels. If, for example, we had two images and one was taken exactly 1 pixels width to the right of the other then we see that both images will contain the same amount of information, in effect we can generate one using the other. However, if the second image is known to be taken half a pixels width to the right then there is the potential to extract more information.

Once we have registered our images we then interpolate points at a higher resolution, using a combination of the pixel values from each image at each point. There are a number of different interpolation algorithms that we can use; for example nearest neighbour or (bi)linear interpolation.

We can then take our final outputted image and apply some post-processing effects to reduce the noise or blur. For example, we could apply a Wiener filter\cite{wienerfilter}.

\subsubsection*{Details}
To understand the uncertainty that is introduced we must look more closely at how it works.
To begin with we model the process that happens when a discrete, low resolution image is taken of a continuous scene in the real world. We can express the low resolution images in terms of a perfect, non-aliased, high resolution version which, in an ideal world, we would like to recover. From the Shannon-Nyquist sampling theory an accurate representation is one that is sampled at twice the frequency of the real world signal\cite{shannonnyquist}. In our model we will assume that the resolution of our perfect image is $LN * LN$ and our actual camera takes images at the resolution of $N * N$ and the scale factor between the two is $L$.

There are a number of effects that combine to create the final image during acquisition. Each image may be taken from a different position so we will need to account for translations, rotations or other warping. Then various optical effects, such as the limit of diffraction, aberrations or being out of focus, will cause some degree of blurring. Finally the image will be sampled at the resolution of the camera and a certain amount of noise will be present.

These processes can be represented by a series of transformations that can be applied to our ideal image.

\begin{align*} 
& y = DBMx + n \nonumber \\ 
& \text{where} \\
& y \text{ - low resolution image} \nonumber \\
& x \text{ - high resolution (ideal) image} \nonumber \\
& M \text{ - warping (translation, rotation)} \nonumber \\
& B \text{ - blurring (optical effects, motion blur)} \nonumber \\
& D \text{ - scaling (discrete sampling)} \nonumber \\
& n \text{ - noise} \nonumber
\end{align*}

The task of image super resolution is given a number of these such images $y$, can we reverse the process to find $x$, our high resolution version.

Finding $M$ is a case of estimating the movement between each of the images that we have. Depending on the application finding these translations can be a very difficult task; in the case of MRI we are dealing with a controlled environment and so it is possible to take images with a known relationship to each other. If the target remains stationary, this makes finding the translations straight-forward, however dealing with movement (such as breathing) makes this a significantly harder task.

Finding B requires modelling the optical effects that occur at each pixel. This will vary between devices and is normally modelled by a point spread function (PSF) that is a spatial averaging of the pixel values in the area. The PSF can be assumed to be the same throughout the image, known as linear space invariant (LSI), or can vary across the image, known as linear space variant (LSV). An example where a linear space variant PSF would be suitable is where there is more heavy blurring near the edge of an image.

The matrix $D$ then samples the result to produce a lower resolution, aliased image and then some noise, $n$, is added to model random variation in the sensor.

These matrices can be combined to give us this result, without loss of generality:

\begin{align}
& y = Wx + n \nonumber
\end{align}

\chapter{Project Plan}

\chapter{Evaluation}

\bibliography{bibliography} 
\bibliographystyle{plain}
    
\end{document}
