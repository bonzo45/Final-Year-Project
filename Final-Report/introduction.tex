\chapter{Introduction}\label{chapter:introduction}

% Provide necessary background.
% Why is the project important?

Magnetic Resonance Imaging (MRI) has found itself being used in an ever increasing number of applications. Uses range from imaging the spinal cord and heart to detecting brain activity (fMRI) and refinements to the technique have resulted in its use in fetal imaging increasing.

In many ways MRI is highly suited to fetal imaging; it has impressive contrast in soft tissue and no known harmful side effects. However, limitations in resolution make imaging what can often be a very small target difficult. Super-resolution reconstruction is a technique designed to mitigate this problem by constructing a high resolution volume from a number of lower resolution scans.

Of course the quality of the reconstruction depends on the quality of these scans. If, for example, the fetus moves during the scanning process, then some areas will be less well sampled than others. This results in less data to use in some areas of the reconstruction than others and gives rise to areas of high uncertainty.
 
From just looking at the reconstructed image it is not necessarily obvious where these areas are. Crucially when using these images for diagnostic purposes it very is important to be aware of this. If a diagnosis needs to be made based on a region of the reconstruction with high uncertainty then this may indicate that further scanning.

% Specify main objectives.
% Specify main contributions - provide section numbers.

The main aim of the project was to prototype visualizations to communicate this uncertainty to researchers and radiographers. This will allow them to both gauge how well a reconstruction went and highlight any particular problem areas.

Two distinctive approaches have been prototyped and their effectiveness has been evaluated with a group of expert users. The first technique (section \ref{section:thresholding}) uses thresholding to find areas of the scan within a specific range of uncertainty; these areas can then be overlayed on the scan in 2D or displayed as a volume in 3D. The second technique (section \ref{section:uncertaintysurface}) projects the uncertainty to a surface representation of the object that has been scanned.

The trade-offs between the two approaches have been studied to establish what works well and where future development should be focused. See chapter \ref{chapter:evaluation}.

As well as being able to visualize the uncertainty, recent research\cite{uncertaintysvd} has shown that it is possible to use the uncertainty to guide the scanning process. The idea is if we know, at the time of scanning, the areas of high uncertainty then we can then tailor the next scan position and direction to target these problem areas. A visualization designed to communicate this information has also been created (see section \ref{section:nextscanplane}) and evaluated.

Finally these visualizations have been wrapped up in a research tool that combines every stage in the reconstruction pipeline together. It allows the user to simulate a scan (section \ref{section:simulatescan}), reconstruct the image (section \ref{section:reconstruction}) and then visualize the results.

The scan simulation feature allows an existing, well reconstructed image to be used as the basis for creating new, lower resolution scans. The idea is that we can assume that the initial image is the 'perfect' reconstruction, create lower resolution scans from it, run our reconstruction algorithm and then compare the result to the original. This will allow different reconstruction algorithms to be thoroughly tested without having to scan thousands of patients. Currently a simple motion corruption simulation can be enabled which will allow the tolerance of a reconstruction algorithm to motion to be tested. Similarly this feature has been evaluated with expert users. See chapter \ref{chapter:evaluation}.

The reconstruction feature provides an easy interface to the fast GPU (Graphics Processing Unit) reconstruction implementation from a previous research paper\cite{gpureconstruction}. This code takes a set of motion corrupted slice stacks (scans) and creates a single reconstructed volume by using SVR (slice-to-volume reconstruction) techniques. 

Additionally one way to guide the reconstruction is by marking each slice stack with a set of landmarks (e.g. Left Eye, Right Eye, ...); this helps the registration step align each of the stacks. A prototype to allow these landmarks to be marked on each slice stack before reconstruction has been created.