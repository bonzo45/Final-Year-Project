% The abstract is a very brief summary of the report's contents. 
% It should be about half a page long. 
% Somebody unfamiliar with your project should have a good idea of what it's about having read the abstract alone and will know whether it will be of interest to them. 
% Note that the abstract is a summary of the entire project including its conclusions. 
% A common mistake is to provide only introductory elements in the abstract without saying what has been achieved.

\chapter*{Abstract}
This project aims to combine all stages of the fetal MRI motion-correction and reconstruction pipeline into an easy-to-use tool. The tool has three main parts: scan simulation, reconstruction and visualization.

Scans are simulated by resampling a previous, 'perfect', reconstruction. These simulations, which include optional motion corruption, are then used to test the performance of a reconstruction algorithm.

Reconstruction using fast GPU reconstruction code, previously only accessible via the command line, is included. In addition landmarks can be provided by the user to guide the reconstruction.

Two visualization techniques for examining the uncertainty in the reconstructed volume are compared. The first uses volume rendering and the second maps uncertainty to a surface. A third visualization to show where to scan next to target uncertain regions is also presented.

Comments from an expert user trial are promising. Including all the functionality in one tool is a step forward and makes the pipeline easier to manage. Scan simulation has a solid foundation however the simulation of additional artefacts will need to be implemented in future work. Providing landmarks received a mixed response with most willing to use it but the time required needs to be reduced.

Using volume rendering for uncertainty visualization is the most popular as it provides finer control and detail. However, the surface mapping approach is found to provide a good overview of the uncertainty. Feedback for the next scan visualization is positive, however further optimization of the reconstruction algorithm is required before it can be used in practice.